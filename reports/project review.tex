%%% LaTeX Template: Two column assignment for BRSU
%%% Based on two column article from: http://www.howtotex.com/
%%% Preamble
\documentclass[	DIV=calc,%
				paper=a4,%
				fontsize=11pt,%
				twocolumn]{scrartcl}	 % KOMA-article class

\usepackage{lipsum}	% Package to create dummy text
\usepackage{blindtext}
\usepackage[english]{babel}	                          % English language/hyphenation
\usepackage[protrusion=true,expansion=true]{microtype} % Better typography
\usepackage{amsmath,amsfonts,amsthm}					 % Math packages
\usepackage[pdftex]{graphicx}	                          % Enable pdflatex
\usepackage[svgnames]{xcolor}	                          % Enabling colors by their 'svgnames'
\usepackage[hang, small,labelfont=bf,up,textfont=it,up]{caption} % Custom captions under/above floats
\usepackage{epstopdf}	 % Converts .eps to .pdf
\usepackage{subfig}	     % Subfigures
\usepackage{booktabs}	 % Nicer tables
\usepackage{fix-cm}       % Custom fontsizes
\usepackage{listings}
\usepackage{soul}

%%% Custom sectioning (sectsty package)
\usepackage{sectsty}	 % Custom sectioning (see below)
\allsectionsfont{%% Change font of al section commands
	\usefont{OT1}{phv}{b}{n}%% bch-b-n: CharterBT-Bold font
	}

\sectionfont{%% Change font of \section command
	\usefont{OT1}{phv}{b}{n}%% bch-b-n: CharterBT-Bold font
	}


\definecolor{brsugrey}{rgb}{0.9, 0.9, 0.9}
\definecolor{brsublue}{rgb}{0, 0.594, 0.949}


\newcommand{\upperRomannumeral}[1]{\uppercase\expandafter{\romannumeral#1}}

%%% Headers and footers
\usepackage{fancyhdr} % Needed to define custom headers/footers
	\pagestyle{fancy} % Enabling the custom headers/footers
\usepackage{lastpage}	

% Header (empty)
\lhead{}
\chead{}
\rhead{}
% Footer (you may change this to your own needs)
\lfoot{\footnotesize 
\texttt{LAA} % Set to the course abbreviation 
\textbullet ~ Quignon % Set to your name
\textbullet ~ Review 2014 LA Project Reports} % Set the assignment number
\cfoot{}
\rfoot{\footnotesize page \thepage\ of \pageref{LastPage}}	% "Page 1 of 2"
\renewcommand{\headrulewidth}{0.0pt}
\renewcommand{\footrulewidth}{0.4pt}



%%% Creating an initial of the very first character of the content
\usepackage{lettrine}
\newcommand{\initial}[1]{%
     \lettrine[lines=3,lhang=0.3,nindent=0em]{
     				\color{brsublue}
     				{\textsf{#1}}}{}}

%%% Title, author and date metadata
\usepackage{titling}	% For custom titles

\newcommand{\HorRule}{\color{brsublue}% Creating a horizontal rule
					 \rule{\linewidth}{1pt}%
					 \color{black}
					 }

\pretitle{\vspace{-30pt} \begin{flushleft} \HorRule 
				\fontsize{25}{25} \usefont{OT1}{phv}{b}{n} \color{gray} \selectfont 
				}
\title{Learning and Adaptivity
\\ Review 2014 LA Project Reports}% Title of your article goes here
\posttitle{\par\end{flushleft}\vskip 0.5em}

\preauthor{\begin{flushleft}
\large \lineskip 0.25em \usefont{OT1}{phv}{b}{sl} \color{brsublue}}
\author{Christophe Quignon }	% Author name goes here
\postauthor{\footnotesize \usefont{OT1}{phv}{m}{sl} \color{Black} 
BRS University of Applied Sciences % Institution of author
\\email: cuignon@gmail.com ~github: @ChrisQuignon
\par\end{flushleft}\HorRule}

\date{\today} 

%%% Begin document
\begin{document}
\maketitle
\thispagestyle{fancy} % Enabling the custom headers/footers for the first page 
\section{Time Series Prediction (Stock Market Prediction)}
Saugata Biswas, Ashok Meenakshi Sundaram

% The first character should be within \initial{} - yes, should...
\begin{itemize}
\item The goal (prediction is related to my project, so I choose this as a first report to review
\item The data was standard stock data
\item A split of 70/15/15 percent for learning, prediction and validation seems a good choice, but 538 entries are not that much.
\item The NARX model is new to me but seems to be good choice.
\item A reason to why this model was choosen is missing.
\item The use of SVMs for regression is interesting but an explanation of the method in detail is missing
\item The results are good, even so the prediction is more conseratie then the actual data
\item Th error histograms are hard to read and interpret
\item An overlay of the predictions would have been nice for comparison.\
\item The given errors help to interpret the results
\item The conclusion is quite short and does not address possible issues but only compares the error values. 
\end{itemize}

\section{Gender Recognition using Facial Key
Points
}
Nicolas Lavderde Alfonso

\begin{itemize}
\item Gender recognition in images seems to be a hard problem
\item The introduction in quite general and not problem related
\item Item the use of standard Databases makes comparison easy, good choice
\item Manual labelling 1521 images, who exactly did this?
\item The facial key points where also labelled.
\item I miss a description of an automatic labelling technique
\item It is not clear whether the algorithm uses the image or the labels
\item  Normalization - good choice, but why?
\item The histograms are a good idea, but I disagree on the separability\
\item A justification for the number of layer and neurons is missing
\item Again a 70/30 ratio for test and training
\item The face classification error looks suspicious, which is not addressed
\item The error is quite fluid, to speak of an optimal parameter set is bold for sure
\item I would disagree on how it is easy to see the improvement
\item The results miss a comparison to other results on the dataset
\item The correlation analysis is a good and interesting point
\item The correlation could also be seen in the histograms
\item The custom NN is an interesting idea, but I do not get why this should lead to something
\item Criticism of the dataset should not be in the conclusion but in the beginning
\item Controlled conditions are a nice thing to increase accuracy but does not account to your learning technique
\item Parameter relation is something the NN should learn on itself
\end{itemize}



\end{document}