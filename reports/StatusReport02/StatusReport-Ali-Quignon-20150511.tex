%\documentclass{beamer}
%\usetheme{Pittsburgh}
\documentclass{scrartcl}

\usepackage[utf8]{inputenc}
\usepackage{default}
\usepackage[procnames]{listings}
\usepackage{graphicx}
%\usepackage[toc,page]{appendix}
\usepackage{caption}
\usepackage{hyperref}
\usepackage{color}
%\usepackage{csvsimple}
\usepackage{float}
%\usepackage[T1]{fontenc}



%Bibliogrpahy?
%\usepackage{bibentry}
%\nobibliography*
%\bibentry{ }


%Python
\definecolor{keywords}{RGB}{255,0,90}
\definecolor{comments}{RGB}{0,0,113}
\definecolor{red}{RGB}{160,0,0}
\definecolor{green}{RGB}{0,150,0}
\lstset{language=Python,
    basicstyle=\ttfamily\scriptsize,
    keywordstyle=\color{keywords},
    commentstyle=\color{comments},
    stringstyle=\color{red},
    identifierstyle=\color{green},
    breaklines = true,
    columns=fullflexible,
    %Numbering and tabs
    %numbers=left,
    %numberstyle=\tiny\color{gray},
    %stepnumber=2,
    %numbersep=1em,
    tabsize=4,
    showspaces=false,
    showstringspaces=false}

\begin{document}

\title{Learning and Adaptivity}
\subtitle{Report No. 2}
\author{
  \href{daiem.ali@smail.inf.h-brs.de}{Ali, Daiem}: \href{https://github.com/daiemna}{github.com/daiemna}\\
  \href{christophe.quignon@smail.inf.h-brs.de}{Quignon, Christophe}:\href{https://github.com/ChrisQuignon}{github.com/ChrisQuignon}
  %Familyname, Name
}
\date{\today}


\maketitle

%TODO: add abstract and conclusion
%labels (or zero)
%references (or zero)
%remove scaffolding code


\begin{abstract}
%TODO :review this!
\textbf{Abstract:}
\end{abstract}


\section{Project introduction}
%Include an introduction section to your project

Heat pumps are a sustainable way to transfer thermal energy into out away from builds to keep a comfortable temperature. But to operate a heating pump is not a trivial task, different building distribute the heat differently and weather with its chaotic nature has a major influence on the temperature flow. In addition, they suffer from a suboptimal efficiency because they often have a bad time delay from sensing to acting. This could be counteracted by predicting future temperatures to overact sensors. efficiency could be increased by predicting energy consumption and delay that to times where energy is cheap. 
Thus we want to predict the behaviour or an energy pump with respect to the weather.


\section{Time Series Prediction}
% Include information on which features are available from your data, and include visualizations

\begin{figure}[H]
  \centering
  \includegraphics[width=1\linewidth]{img/hourly_time_delta_test.png}
  \caption[This is needed]{RMSE of time series prediction where time ranges varies for t+1h,t+2h ... t+100h where h=hours.\footnotemark}
  \label{fig:hourly_time_delta}
\end{figure}


\begin{figure}[H]
  \centering
  \includegraphics[width=1\linewidth]{img/daily_time_delta_test.png}
  \caption[This is needed]{RMSE of time series prediction where time ranges varies for t+1d,t+2d ... t+60d where d=days.\footnotemark}
  \label{fig:hourly_time_delta}
\end{figure}

\section{Method}

\section{Conclusion} 
%TODO
%http://blog.kaggle.com/2012/05/01/chucking-everything-into-a-random-forest-ben-hamner-on-winning-the-air-quality-prediction-hackathon/

%BIBLIOGRPAHY?
\bibliographystyle{plain}%amsalpha
\bibliography{bib.bib}
%\bibentry{}

%\begin{appendix}
%\section{}

%\end{appendix}


%COPY AND PASTE FROM HERE

%\begin{enumerate}
% \item
%\end{enumerate}

%\href{link}{text}

%\begin[Language=Python]{lstlisting}
%#PYTHON CODE HERE
%\end{lstlisting}

%\lstinputlisting[language=Python]{	}

%\csvautotabular[separator=semicolon]{data.csv}

%\subsubsection{left}
%\begin{figure}[H]
%  \centering
%  \includegraphics[width=0.5\linewidth]{../img/	}
%  %\caption{}
%  %\label{fig:}
%\end{figure}
%PUT UNITS ON THE FIGURES

\end{document}
